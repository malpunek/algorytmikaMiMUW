\chapter{Przepływy}

\begin{defi}[Sieć przepływowa]
    Siecią przepływową nazywamy N = (G,c,s,t), gdzie:
    \begin{enumerate}
        \item $G = (V,E)$
        \item $c = E \rightarrow \RR_+$ przy czym możemy pisać jakby było $V \times V \rightarrow \RR_+$ 
        \item $s,t \in V$
    \end{enumerate}
\end{defi}

\begin{defi}[Przepływ]
    Przepływem w sieci przepływowej N nazywamy funkcję $f: V \times V \rightarrow R$ taką, że:

    \begin{enumerate}
        \item $\forall_{(u,v) \in E} |f(u,v)| \le c(u,v)$
        \item $\forall_{(u,v) \in E} f(u,v) =- f(u,v)$
        \item $\forall_{(v) \in E\setminus \{s,t\}}  \sum_{u \in E} f(u,v) = 0$
    \end{enumerate}
\end{defi}

\begin{defi}[Warotść przepływu]
    Wartością przepływu nazywamy $|f| = \sum_v f(s,v)$
\end{defi}

Będziemy analizowali algorytmy szukamjące przepływu o największej wartości.

Uwaga: Od tej pory zakładamy że $\exists_{(u,v) \in E} \Rightarrow \hspace{3pt}!\exists_{(v,u) \in E}$. \newline Jeśli jest inaczej to zawsze możemy stworzyć nowy wierzchołek $(uv)$ i krawędź $ u \rightarrow v $ zastąpić krawędziami $ u \rightarrow (uv) \rightarrow v$ z odpowiednimi wagami. Równoważność tych dwóch sieci przepływowych jest oczywista.

\section{Algorytm zachłanny - Forda-Fulkersona}
Szkic:

\begin{algorithmic}
    \State $f:= 0  \hspace{2cm} (\iff \forall_{u,v} f(u,v) = 0)$
    \ForAll {$i \in E$}
    \State powiększaj $f$ wzdłuż ścieżek od $s$ do $t$
    \EndFor
\end{algorithmic}

\begin{defi}[Sieć residualna]
    Sieć residualna dla przepływu f w N to sieć $N_f = (G = \{V, E_f\}, c_f, s, t)$, gdzie:
    \begin{itemize}
        \item $c_f(u, v) = c(u,v) - f(u,v)$ (przepustowość residualna)
        \item $E_f = \{ (u,v) \in V^2 : c_f(u,c) > 0 \}$
    \end{itemize}
\end{defi}

\begin{defi}[Ścieżka powiększająca]
    Ścieżka powiększająca to dowolna ścieżka $s \rightarrow t \in G_f$
\end{defi}

Niech będą dane przepływ f, ścieżka powiększająca P.

Niech $f_p: V^2 \rightarrow R$ \newline
$f_p(u,v) =
\begin{cases}
    min_{(x,y) \in P} c_f(x,y) \hspace{2cm} \text{gdy} (u,v) \in P \\
    min_{(x,y) \in P} c_f(x,y) \hspace{2cm} \text{gdy} (v,u) \in P \\
    0 \hspace{2cm} \text{wpp}
\end{cases}$


Note: $f_p$ jest przepływem w $N_f$

Będziemy używać dodawania funkcji: $(f + f_p)(x) = f(x) + f_p(x)$

Algorytm FF:

\begin{algorithmic}
    \State $f \gets 0$
    \While {$\exists P$}
        \State $f := f + f_p$
        \State uaktualnij $N_f$
        \EndWhile
\end{algorithmic}

Czasy:
\begin{itemize}[noitemsep, topsep=0pt]
    \item istnienie P sprawdzamy w O($G_f$) za pomocą DFS-a
    \item uaktualnianie $N_f$ w O(P)
\end{itemize}

$|f^*|$ - maksymalny przepływ

Jeśli przepustowości są całkowite $ \iff c: V^2 \rightarrow \NN$, wtedy mamy $\le |f^*|$ iteracji (bo w każdej iteracji $|f|$ się zwiększa), zatem czas $ \le O(|f^*|E)$  

\subsection{Dowód poprawności}

\begin{defi}[Przekrój]
    Przekrojem w sieci $N = (G,c,s,t)$ nazywamy dowolny podział $V$ na zbiory S, T takie że $S \cup T = V \land S \cap T = \emptyset$
\end{defi}

\begin{defi}[Przepustowość przekroju i przepływ przez przekrój]
    Przepustowością przekroju nazywamy $c(S,T) = \sum_{u \in S, v \in T} c(u,v)$, zaś przepływem przez przekrój $f(S,T) = \sum_{u \in S, v \in T} f(u,v)$
\end{defi}

\begin{lemat}
    f przepływ, P ścieżka powiększająca $\Rightarrow f + f_p$ jest przepływem
\end{lemat}

\begin{lemat}
    f przepływ w N, g przepływ w $N_f \Rightarrow (f + g)$ jest przepływem w N
\end{lemat}

\begin{lemat}
    $f(S,T) \le c(S,T)$ dowód oczywisty na podstawie definicji.
\end{lemat}

\begin{lemat}
    $\forall_{(S,T)} f(S,T) = |f| = f(\{ s \}, V \setminus \{ s \} ) = \sum_{v \in V} f(s,v)$
\end{lemat}
\begin{proof}
     $f(S,T) = \sum_{u \in S, v \in T} f(u,v) = \sum_{u \in S, v \in T} f(u,v) - \sum_{u \in S, v \in S} f(u,v) $ wiemy, że $\sum_{u \in S, v \in S} f(u,v) = 0$ zatem to dalej jest równe $ = \sum_{v \in V} f(s,v) + \sum_{u \in S \setminus \{ s \}, v \in V } f(u,v)$ przy czym jak drugi składnik rozbijemy na $\sum \sum$ to z warunku (3) z definicji przepływu = 0. zatem to dalej jest równe $ |f| $
\end{proof}


Po co nam to?
Jak połączymy te dwie nierówności to mamy:
$\forall_f \forall_{S, T} |f| \le c(S,T)$

Chcąc udowodnić że algorytm tworzy maksymalny przepływ pokażemy że powyżej zajdzie równość.

Twierdzenie o maksymalnym przepływie i minimalnym przekroju - jeśli f to przepływ w N to są równoważne:
\begin{enumerate}
    \item f jest maksymalny
    \item nie istnieje ścieżka powiększająca
    \item istnieje przekrót (S,T) taki, że $ c(S,T) = |f|$
    
\end{enumerate}

Dowód:
\begin{itemize}
\item $1 \Rightarrow 2$ bo gdyby $ \exists P \Rightarrow f + f_p > f$
\item $3 \Rightarrow 1$ z czegoś wyżej
\item $2 \Rightarrow 3$ Buduję przekrój $ S:= $ wierzchołki osiągalne z $s w G_f. T:= V \setminus S$. To jest przekrój. $|f| = \sum_{u \in S, v \in T} f(u,v) = \sum_{u \in S, v \in T} c(u,v)$. Ostatnia równość zachodzi bo 2. 
\end{itemize}

Zatem jeśli algorytm się zatrzyma, to nie istnieje ścieżka powiększająca, zatem wyznacza maksymalny przepływ. Działa też dla wymiernych, bo można pomnożyć razy wszystkie mianowniki.

\section{Algorytm Edmondsa-Karpa}

Spróbujemy wybrać jakoś ścieżkę powiększającą w algorytmie Ford-Fulkersona, żeby to optymalizować.

Pierwszy pomysł: weźmy najkrótsze ścieżki (w sensie liczby krawędzi). Najkrótszą ścieżkę będziemy wyszukiwać BFS-em.

Poprawność jest oczywista na podstawie poprawności Forda-Fulkersona.

Zastanówmy się nad złożonością.

Iteracja dalej trwa $O(E)$.
Spróbujemy znaleźć ograniczenie na liczbę iteracji.
Będziemy pisać odległość $s \rightarrow v$ w $G_{f}$ jako $\delta_{f}(s,v)$
Zaczniemy od pokazania:

\begin{lemat}
    $f$ - przepływ, $P$ - najkrótsza ścieżka powiększająca, $f' = f + f_p \Rightarrow \forall_{v \in V} \delta_{f'}(s,v) \ge \delta_{f}(s,v)$
\end{lemat}

% TODO popraw ten dowód

\begin{proof}
Indukcja:

jeśli $(w,v) \in E_f$ to $\delta_{f'}(v) = \delta_{f'}(w) + 1 \ge \delta_f(w) + 1 \ge \delta_f(v)$

$(w,v) \notin E_f \Rightarrow (v,w) \in E_f \Rightarrow \delta_{f'}(w) = \delta_f(v) + 1 \Rightarrow \delta_{f'}(v) \ge \delta_f(w) + 1 = \delta_f(v)+2$
\end{proof}

\begin{lemat}
    W algorytmie Edmondsa-Karpa każda krawędź staje się nasycona $O(V)$ razy.
\end{lemat}

Z tego jasno wyniknie że liczba iteracji jest $O(V*E)$ (przy każdej iteracji conajmniej jedna krawędź staje się nasycona). (Dopiero teraz pokazaliśmy, że problem maksymalnego przepływu można rozwiązać wielomianowo)

\begin{proof}
    Weźmy sobie $e = (u,v) \in E_f$. Spójrzmy na dwa kolejne momenty A, B, kiedy $e$ stawała się nasycona.
    W momentach A,B $e \in E_f$. Ale po momencie A mamy $e \notin A$. Czyli kiedyś musiała się pojawić. Nazwijmy ten moment C.
    
    Oczywiście $(v,u) \in E_f$ w momencie C i $(v,u)$ leży na najkrótszej ścieżce $s \rightarrow t$. Zatem $\delta_{f'}(u) = \delta_{f'}(v)+1$. Z poprzedniego lematu $\ge \delta_f(v) + 1 = \delta_f(u) + 1$

    Pokazaliśmy, że pomiędzy A i B $\delta_f(u)$ rośnie o $\ge 2$, zatem tak się dzieje $\le (V-1)/2$ razy.  
\end{proof}

\section{Algorytm Dinica i trzech hindusów}

Znowu naturalny pomysł. Bezsensu odpalamy BFS-a wiele razy. Być może po updacie da się reużyć wynik naszego ostatniego BFS-a i updateować 'na jedno odpalenie BFS więcej ścieżek'

\begin{defi}[Przepływ blokujący]
    taki, że dla każdej ścieżki $s \rightarrow t$ istnieje krawędź nasycona.
\end{defi}

Będziemy robić mniej-więcej to samo, tylko w jednej iteracji będziemy szukać przepływu blokującego zamiast ścieżki powiększającej (blokującej).

\begin{defi}[Sieć warstwowa]
    $L_f$ to jest pewien podgraf $N_f$ taki, że:
    \begin{itemize}
        \item $V(L_f) = V$ 
        \item $(u,v) \in E(L_f) \iff \delta_f(v)=\delta_f(u) + 1$
    \end{itemize}
\end{defi}

Schemat:

\begin{algorithmic}
    \State $f \gets 0$
    \While {$\exists$ ścieżka powiększająca}
        \State Zbuduj sieć warstwową $L_f$ za pomocą BFS-a
        \State $b \gets $ przepływ blokujący w $L_f$
        \State $f \gets f + b$
        \State Zaktualizuj $N_f$
        \EndWhile
\end{algorithmic}

Dlaczego to działa szybko? Będziemy ograniczać liczbę iteracji:

\begin{lemat}
    Niech b będzie przepływem blokującym w $L_f$
    $\delta_{f+b}(t) > \delta_f(t)$
\end{lemat}

\begin{proof}
    Niech: $d = \delta_f(t)$ \newline
    Pokażemy że dowolna ścieżka w $G_{f+b}: v_0, v_1, ... v_k$ ma długość $> d$. \newline
    Niech: $l(v_i) = \delta_f(v_i)$ \newline
    Wtedy $\forall_{(v_i,v_{i+1})} \in G_{f+b} \rightarrow (v_i,v_{i+1}) \in G_f \rightarrow \delta_f(v_{i+1} \le \delta_f(v)+1)$
    lub $(v_i,v_{i+1}) \notin G_f \rightarrow b$ przesłał po $(v_{i+1}, v) \rightarrow \delta_f(v_i) = \delta_f(v_{i+1}) + 1 \rightarrow l(v_{i+1}) = l(v_i) -1$

    Co najmniej raz zdarzył się przypadek drugi, bo $b$ jest blokujące.

\end{proof}

Zatem liczba iteracji $\le V-1$.


Pytanie pozostaje, jak szybko potrafimy znaleźć przepływ blokujący.

Będziemy robić dfs-a który jak znajdzie ścieżkę do t to skacze do s, jak się wycofuje to usuwa krawędź.

\begin{algorithmic}
    \State $b \gets 0$
    \State $v \gets s$
    \State $p \gets \emptyset$
    \While {$v != s or \exists_{w} and (v,w \in E(L_f))$}
        \If {$\exists_w (v,w) \in E(L_f)$}
            \State ADVANCE:
            \State $push(p, (v,w))$
            \State $v \gets w$
            \If {$v = t$}
                \State AUGMENT:
                \State Powiększ $b$ wzdłuż ścieżki powiększającej $p$
                \State $p \gets \emptyset$
                \State $ v \gets s $
            \Else
                \State RETREAT:
                \State Usuń ostatnią krawędź $(u,w) z p$
                \State $E(L_f) \gets E(L_f) \setminus {(u,w)}$
                \State $v \gets u$
            \EndIf
        \EndIf
    \EndWhile

\end{algorithmic}

Analiza:
całkowity czas RETREAT = O(E)
całkowity czas AUGMENT = O(VE)
całkowity czas ADVANCE: każda operacja ADVANCE(v,w) w przyszłości owocuje RETREAT(v,w) lub AUGMENT, więc $\le O(VE) + O(E)$

suma: $O(VE)$

Zatem:
całkowity koszt algorytmu Dinica jest równy $O(V^2E)$


Trzech hindusów znajduje przepływ blokujący w czasie $O(V^2) \rightarrow $ cały przepływ w $O(V^3)$. Da się 

