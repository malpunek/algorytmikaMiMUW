\section{Przepływy}


Sieć przepływowa: graf skierowany, każda krawędź etykietowana liczbą z R, source, target
N = (G,c,s,t), gdzie:
\begin{enumerate}

\item $G = (V,E)$
\item $c = E \rightarrow R_+$ przy czym możemy pisać jakby było $VxV \rightarrow R_+$ 
\item $s,t \in V$
\end{enumerate}

Przepływem w grafie N nazywamy funkcję $f: VxV \rightarrow R$

\begin{enumerate}
\item $\forall_{(u,v) \in E} |f(u,v)| \le c(u,v)$
\item $\forall_{(u,v) \in E} f(u,v) =- f(u,v)$
\item $\forall_{(v) \in E\setminus \{s,t\}}  \sum_{u \in E} f(u,v) = 0$
\end{enumerate}

Wartość przepływu: $|f| = \sum_v F(s,v)$

Szukamy przepływu o największej wartości.

Note: zakładamy że $\exists_{(u,v) \in E} \Rightarrow !\exists_{(v,u) \in E} $

\subsection{Algorytm zachłanny - Forda-Fulkersona}
Szkic:

$f:= 0  (\forall_{u,v} f(u,v) = 0$
$foreach(i \in E)$ do
    powiększaj $f$ wzdłuż ścieżek od $s$ do $t$

Sieć residualna dla przepływu f w N to sieć $N_f = (G = \{V, E_f\}, c_f, s, t)$, gdzie
$c_f(u, v) = c(u,v) - f(u,v)$ (przepustowość residualna)
$E_f = \{ (u,v) \in V^2 : c_f(u,c) > 0 \}$

Ścieżka powiększająca to dowolna ścieżka $s \rightarrow t \in G_f$

Niech będą dane przepływ f, ścieżka powiększająca P.

$f_p: V^2 \rightarrow R$
$f_p(u,v) = min_{(x,y) \in P} c_f(x,y) gdy (u,v) \in P;min_{(x,y) \in P} c_f(x,y) gdy (v,u) \in P; else 0$

Note: $f_p$ jest przepływem w $N_f$

Będziemy używać dodawania funkcji: $(f + f_p)(x) = f(x) + f_p(x)$

Algorytm:
    $f \leftarrow 0$
    $while \exists P$
        $f := f + f_p$
        Uaktualnij $N_f$

Czasy:
 - istnienie O($G_f$) za pomocą DFS-a;
 - uaktualnianie $N_f$ w O(P)

Jeśli przepustowości są całkowite $ \iff c: V^2 \rightarrow Nat$ wtedy mamy $\le |f*|$ iteracji (bo w każdej iteracji |f| się zwiększa)
$|f*|$ - maksymalny przepływ

Czyli czas $ \le O(|f*|E)$  

Dowód poprawności:

Przekrojem w sieci $N = (G,c,s,t)$ nazywamy dowolny podział na zbiory S, T takie że $S \cup T = V and S \cap T = emptysetocirc$

przepustowość przekroju $c(S,T) = \sum_{u \in S, v \in T} c(u,v)$

Niech f przepływ w N. (S, T) = przekrój w N
Wtedy liczbę $f(S,T) = \sum_{u \in S, v \in T} f(u,v)$ nazywamy przepływem przez przekrój.

Lemat 0: f przepływ, P ścieżka powiększająca $f + f_p$ jest przepływem 

Lemat 0.5: f przepływ w N, g przepływ w $N_f \Rightarrow (f + g)$ jest przepływem w N
Lemat 1: $f(S,T) \le c(S,T)$ dowód oczywisty na podstawie z definicji.

Lemat 2: $\forall_{(S,T)} f(S,T) = |f| = f(\{ s \}, V \setminus \{ s \} ) = \sum_{v \in V} f(s,v)$
Dowód: $f(S,T) = \sum_{u \in S, v \in T} f(u,v) = \sum_{u \in S, v \in T} f(u,v) - \sum_{u \in S, v \in S} f(u,v) $
wiemy, że $\sum_{u \in S, v \in S} f(u,v) = 0$
zatem to dalej jest równe $ = \sum_{v \in V} f(s,v) + \sum_{u \in S \setminus \{ s \}, v \in V } f(u,v)$
przy czym jak drugi składnik rozbijemy na $\sum \sum$ to z warunku (3) z definicji przepływu = 0.
zatem to dalej jest równe $ |f| $


Po co nam to?
Jak połączymy te dwie nierówności to mamy:
$\forall_f \forall_{S, T} |f| \le c(S,T)$

Chcąc udowodnić że algorytm tworzy maksymalny przepływ pokażemy że powyżej zajdzie równość.

Twierdzenie o maksymalnym przepływie i minimalnym przekroju - jeśli f to przepływ w N to są równoważne:
\begin{enumerate}
    \item f jest maksymalny
    \item nie istnieje ścieżka powiększająca
    \item istnieje przekrót (S,T) taki, że $ c(S,T) = |f|$
    
\end{enumerate}

Dowód: $1 \Rightarrow 2$

